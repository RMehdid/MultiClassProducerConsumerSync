During the implementation of the multi-class producer-consumer synchronization system, several challenges were encountered. In this section, we outline the key challenges faced and discuss the strategies employed to address them.

\section{Challenge 1: Synchronization}

One of the primary challenges was achieving effective synchronization among the multiple processes, ensuring that producers and consumers operate correctly without data inconsistencies.

\textbf{Solution:} We implemented semaphore-based synchronization to control access to the shared memory. This approach helped in preventing race conditions and maintaining the integrity of the data.

\section{Challenge 2: Shared Memory Access}

Managing shared memory access and preventing conflicts between processes accessing the shared buffer posed a significant challenge.

\textbf{Solution:} Careful consideration was given to the order of operations and the use of semaphores to regulate access. This helped in avoiding conflicts and maintaining the class-based constraints.

\section{Challenge 3: Debugging Forked Processes}

Debugging issues related to forked processes, especially in identifying the origin of errors and ensuring proper termination, presented a challenge.

\textbf{Solution:} Extensive logging and debugging statements were incorporated into the code. Additionally, systematic testing and monitoring were employed to trace the behavior of individual processes.