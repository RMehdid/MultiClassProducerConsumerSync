The multi-class producer-consumer synchronization system successfully met the project requirements. Through testing and observation, several key conclusions can be drawn.

\section{Achievements}

\begin{itemize}
    \item The system effectively handled concurrent execution, ensuring proper coordination among producers and consumers.
    \item Stress testing demonstrated the robustness of the system, with no observable issues under high loads.
    \item Class-based constraints were maintained, preventing cross-class interactions, and ensuring proper message handling.
    \item The system adapted seamlessly to reading messages from a file, demonstrating flexibility and expanded functionality.
    \item The system exhibited graceful error handling, responding well to unexpected events without compromising stability.
\end{itemize}

\section{Lessons Learned}

This project provided valuable insights into process synchronization and shared memory management. Key takeaways include:

\begin{itemize}
    \item The importance of using synchronization mechanisms, such as semaphores, to prevent race conditions.
    \item Proper design considerations for shared memory structures to maintain data integrity.
    \item The significance of error handling mechanisms for robust system behavior.
\end{itemize}
