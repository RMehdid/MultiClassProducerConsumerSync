\definecolor{xcodeblue}{rgb}{0, 0.4, 0.9}
\definecolor{xcodered}{rgb}{0.85, 0, 0.05}
\definecolor{xcodegreen}{rgb}{0, 0.5, 0}
\definecolor{xdarkgray}{rgb}{0.1, 0.1, 0.1}
\definecolor{xlightgray}{rgb}{0.3, 0.3, 0.3}

\lstdefinestyle{xcodestyle}{
  language=C,
  numbers=left,  % Show line numbers
  stepnumber=1,
  numbersep=5pt,
  backgroundcolor=\color{xdarkgray},
  showspaces=false,
  showstringspaces=false,
  showtabs=false,
  tabsize=2,
  captionpos=b,
  breaklines=true,
  breakatwhitespace=true,
  title=\lstname,
  keywordstyle=\color{xcodeblue},
  commentstyle=\color{xlightgray},
  stringstyle=\color{xcodegreen},
  basicstyle=\color{white}\ttfamily\small,
  frame=none, % No frame
}

\lstset{style=xcodestyle}

\section{Shared Memory and Data Structures}

For inter-process communication, we utilize shared memory with a message buffer and a class integer for the producer.

\lstinputlisting{codeSamples/typedef.c}

\section{Semaphore Initialization}

\lstinputlisting{codeSamples/semaphore_init.c}

\section{Producer Function}

The producer function now reads messages from a file and generates messages based on the class. It updates the shared memory and utilizes semaphores for synchronization, ensuring mutual exclusion during critical sections.

\lstinputlisting{codeSamples/producer.c}

\section{Consumer Function}

The consumer function consumes messages from the shared buffer based on the class. It also utilizes semaphores to ensure proper synchronization.

\lstinputlisting{codeSamples/consumer.c}

\section{Forking Processes}

\lstinputlisting{codeSamples/fork.c}
